% !TeX encoding = UTF-8
%-----------------------------------
% Define document and include general packages
%-----------------------------------


%Template
\documentclass[12pt,oneside,titlepage,listof=totoc,bibliography=totoc]{scrartcl}
\usepackage[utf8]{inputenc}
\usepackage[ngerman]{babel}
\usepackage[babel,german=quotes]{csquotes}
\usepackage[T1]{fontenc}
\usepackage{fancyhdr}
\usepackage{fancybox}
\usepackage[a4paper, left=4cm, right=2cm, top=2.8cm, bottom=2.3cm]{geometry}
\usepackage{graphicx}
\usepackage{colortbl}
\usepackage{array}
\usepackage{float}      %Positionierung von Abb. und Tabellen mit [H] erzwingen
\usepackage{footnote}
\usepackage{caption}
\usepackage{enumitem}
\usepackage{amssymb}
\usepackage{mathptmx}
\usepackage{amsmath}
\usepackage[table]{xcolor}
\usepackage{pdfpages} 
\usepackage{marvosym}			% Verwendung von Symbolen, z.B. perfektes Eurozeichen
\usepackage[colorlinks=true,linkcolor=black]{hyperref}
\definecolor{darkblack}{rgb}{0,0,0}
\hypersetup{colorlinks=true, breaklinks=true, linkcolor=darkblack, menucolor=darkblack, urlcolor=darkblack}
\fontfamily{ptm}\selectfont
\usepackage{underscore}


%Eigener Input
\pdfminorversion=7
%\usepackage{hyphenat}
\hyphenation{Dev-Ops}
\usepackage{setspace}
%\onehalfspacing
%\linespread{1.5}

% Mehrere Fussnoten nacheinander mit Komma separiert
\usepackage[hang, multiple]{footmisc}
\setlength{\footnotemargin}{1em}

\usepackage[iso,german]{isodate}
\usepackage{ifthen}


\usepackage{silence}
\WarningFilter{scrbook}{Usage of package `fancyhdr'}

% todo Aufgaben als Kommentare verfassen für verschiedene Editoren
\usepackage{todonotes}


%Bildüberschrift vergrößern
\usepackage{graphicx}
\usepackage{caption}
\captionsetup[table]{labelfont=bf, textfont=bf}
\captionsetup[figure]{labelfont=bf, textfont=bf}
\captionsetup{justification=raggedright,singlelinecheck=false}
\usepackage[export]{adjustbox}


%Pakete für Tabellen
\usepackage{epstopdf}
\usepackage{nicefrac} % Brüche
\usepackage{multirow}
\usepackage{rotating} % vertikal schreiben
\usepackage{colortbl}
\usepackage{mdwlist}

\definecolor{dunkelgrau}{rgb}{0.8,0.8,0.8}
\definecolor{hellgrau}{rgb}{0.0,0.7,0.99}
% Colors for listings
\definecolor{mauve}{rgb}{0.58,0,0.82}
\definecolor{dkgreen}{rgb}{0,0.6,0}

% sauber formatierter Quelltext
\usepackage{listings}
\lstset{numbers=left,
	numberstyle=\tiny,
	numbersep=5pt,
	breaklines=true,
	showstringspaces=false,
	frame=l ,
	xleftmargin=5pt,
	xrightmargin=5pt,
	basicstyle=\ttfamily\scriptsize,
	stepnumber=1,
	keywordstyle=\color{blue},          % keyword style
  	commentstyle=\color{dkgreen},       % comment style
  	stringstyle=\color{mauve}         % string literal style
}

% Biblatex
\usepackage[
backend=biber,
%style=footnote-dw,
style=verbose-inote,
citestyle=authoryear,
url=false,
isbn=false,
notetype=footonly,
hyperref=false,
sortlocale=de,
firstinits=true,
mincrossrefs=1,
alldates=edtf,
datezeros=true,
dateabbrev=true,
%datezeros=true
]{biblatex}

%weitere Anpassungen für BibLaTex
%Eigene Anpassungen

\addtokomafont{disposition}{\rmfamily} % Angleichung der Schriftart bei den Überschriften

\DeclareLabeldate{% 
	\field{date} 
	\field{year} 
	\field{eventdate} 
	\field{origdate} 
	\literal{nodate} 
}

\DefineBibliographyStrings{german}{nodate = {{}keine Datumsangabe},}

%-------------------

% Opptionen für Biblatex
\ExecuteBibliographyOptions{%
giveninits=false,
isbn=true, 
url=true, 
doi=false, 
eprint=false,
maxbibnames=7, % Alle Autoren (kein et al.)
maxcitenames=2, % et al. ab dem 3. Autor
backref=false, % Rückverweise auf Zitatseiten
bibencoding=utf8, % wenn .bib in utf8, sonst ascii
bibwarn=true, % Warnung bei fehlerhafter bib-Datei
}%

% et al. an Stelle von u.a.
\DefineBibliographyStrings{ngerman}{ 
   andothers = {{et\,al\adddot}},             
}

% Klammern um das Jahr in der Fußnote
\renewbibmacro*{cite:labelyear+extrayear}{% 
  \iffieldundef{labelyear} 
    {} 
    {\printtext[bibhyperref]{% 
       \mkbibparens{% 
         \printfield{labelyear}% 
         \printfield{extrayear}}}}}


%Hier wird das URL vor der url-Adresse entfernt.
\DeclareFieldFormat{url}{\url{#1}}


%Hier wird der kursive booktitle entfernt
\DeclareFieldFormat{booktitle}{{#1}}

%Hier wird der kursive title entfernt
\DeclareFieldFormat[incollection]{title}{{#1}}

%Hier wird der kursive title entfernt
\DeclareFieldFormat[article]{journaltitle}{{#1}}

%Hier wird der kursive title entfernt
\DeclareFieldFormat[article]{title}{{#1}}


\renewcommand*{\multinamedelim}{\addcomma\space}
\renewcommand*{\finalnamedelim}{\addcomma\space}

\DeclareNameFormat{last-first}{%
%	\renewcommand*{\multinamedelim}{\addsemicolon\addspace}%
  	\ifgiveninits
    {\usebibmacro{name:family-given}
        {\namepartfamily}
        {\namepartgiveni}
        {\namepartprefix}
        {\namepartsuffix}
    }
    {\usebibmacro{name:family-given}
        {\namepartfamily}
        {\namepartgiveni}
        {\namepartprefix}
        {\namepartsuffix}
    }%
   \usebibmacro{name:andothers}}

\DeclareNameAlias{author}{last-first} 

\DeclareNameAlias{editor}{last-first} 

\DeclareNameAlias{cite-editor}{last-first} 

% Alternative Notation der Fußnoten 
% Zeigt sowohl den Nachnamen als auch den Vornamen an
% Beispiel: \fullfootcite[Vgl. ][Seite 5]{Tanenbaum.2003} 
\DeclareCiteCommand{\fullfootcite}[\mkbibfootnote] 
  {\usebibmacro{prenote}}                                 
  {\usebibmacro{citeindex}%
  	\setunit{\addnbspace}%
  	\printnames{author}%
  	{\addcomma\space}%
  	\printfield{shorttitle}%
  	\setunit*{,\space}%
  	\printfield{year}
  }
  {\addsemicolon\space}
  {\usebibmacro{postnote}}
  
  %Autoren (Nachname, Vorname)
  \DeclareNameAlias{default}{family-given}
  
% Alternative Notation der Fußnoten 
% Zeigt sowohl den Nachnamen als auch den Vornamen an
% Beispiel: \fullfootcite[Vgl. ][Seite 5]{Tanenbaum.2003} 
\DeclareCiteCommand{\blogfootcite}[\mkbibfootnote]
  {\usebibmacro{prenote}}
  {\usebibmacro{citeindex}%
  	\printnames{author}%
  	\newunit\addcomma\space
  	\addcomma \addspace \printfield{shorttitle}%
  	\addcomma \addspace \printfield{year}}
  \setunit{}
  
  
% Alternative Notation der Fußnoten 
% Zeigt sowohl den Nachnamen als auch den Vornamen an
% Beispiel: \fullfootcite[Vgl. ][Seite 5]{Tanenbaum.2003} 
\DeclareCiteCommand{\interviewcite}[\mkbibfootnote]
	{\usebibmacro{prenote}}
	{\usebibmacro{citeindex}%
		\printnames{author}%
		\newunit\addcomma\space
		\addcomma \addspace \printfield{keywords}%
		\addcomma \addspace \printfield{year}%
		\addcomma \addspace \printfield{edition}}
	{\addsemicolon\space}
	{\usebibmacro{postnote}}
  
  
  
% Alternative Notation der Fußnoten 
% Zeigt sowohl den Nachnamen als auch den Vornamen an
% Beispiel: \fullfootcite[Vgl. ][Seite 5]{Tanenbaum.2003} 
\DeclareCiteCommand{\ifootcite}[\mkbibfootnote]
  {\iffieldundef{prenote}%
	{\usebibmacro{prenote}}%
	\setunit{}}%
  {\usebibmacro{citeindex}%
	\printnames[sortname][1-1]{author}%
	\addcomma \addspace \printfield{keywords}%
	\addcomma \addspace \printfield{year}}
  {\iffieldundef{postnote}%
	{}%
	{\usebibmacro{postnote}\addspace}%
	\setunit{}}%
{\adddot}


%Autoren (Nachname, Vorname)
\DeclareNameAlias{default}{family-given}

%Reihenfolge von publisher, year, address verändern
% Achtung, bisher nur für den Typ @book definiert

%\renewcommand*{\newunitpunct}{\addcomma\space}

%% Definiert @Book Eintrag
\DeclareBibliographyDriver{book}{%
  \printnames{author}\space% 
  (\printfield{shorttitle} \addcomma\space \printfield{year})%
  \setunit{\addcolon\space}
  \printfield{title}%
  \setunit*{,\space}%
  \printfield{edition}%
  \setunit*{\addcomma\space}%
  \newunit\newblockpunct
  \printlist{location}%
  \setunit*{\addcolon \space}%
  \printlist{publisher}
  \setunit*{\addcomma\space}
  \printfield{year}%
  }

%% Definiert @Online Eintrag
\DeclareBibliographyDriver{online}{%
  \printnames{author} \space%
  (\printfield{shorttitle} \addcomma\space %
	\printfield{year})%
  \setunit{\addcolon\space}
  \printfield{title}\space%
  %\newunit\newblock
 (\iffieldundef{month}%
 	{keine Datumsangabe}%
 	{{\isodate \printdate{}}%
 	})
  \newunit\newblock
  \printfield{url} \addcomma\space%
  (\printfield{note})%
  }

%% Definiert @Online Eintrag
\DeclareBibliographyDriver{misc}{%
	\printnames{author}%
	\setunit{\addcolon\space}
	\printfield{title}%
	\setunit*{,\space}%
	%\newunit\newblock
	\printfield{url}%
	\printfield{note}%
	\finentry}
  
%% Definiert @Article Eintrag
\DeclareBibliographyDriver{article}{%
  \printnames{author}\space%
  (\printfield{shorttitle} \addcomma\space \printfield{year})%
  \setunit{\addcolon\space}
  \printfield{title}%
  \setunit*{\addcomma\space in:\space}%
  %\newunit\newblock
  \usebibmacro{journal}\addcomma\space%
  \printfield{volume}%
  \setunit*{\space(}%
  \printfield{year}\newunit{)}%
  \setunit*{\addcomma\space Nr.\space}%
  \printfield{number}
  \setunit*{\addcomma\space}%
  \printfield{pages}}  


%% Definiert @Collection Eintrag
\DeclareBibliographyDriver{collection}{%
	\printnames{editor}%
	\newunit\space
	(Hrsg.)\space%
	(\printfield{shorttitle}\addcomma\space
	\printfield{year})%
	\setunit{\addcolon\space}
	\printfield{title}%
	\setunit*{,\space}%
	\printfield{edition}%
	\setunit*{\addcomma\space}%
	\printlist{publisher} \addcolon
	\newunit\newblockpunct
	\printlist{location}%
	\setunit*{\addcomma \space}%
	\printfield{year}%
	}

%% Definiert @Collection Eintrag
\DeclareBibliographyDriver{incollection}{%
	\printnames{author}%
	(\printfield{shorttitle} \addcomma\space \printfield{year})%
	\setunit{\addcolon\space}
	\printfield{title}%
  	\setunit*{\addcomma\space in:\space}%
	\printnames{editor}%
	\newunit\space(Hrsg.),%
	\setunit*{\space}
	\printfield{booktitle}%
	\setunit*{\space}%
	\printfield{year}%
	\setunit*{,\space}% 
  	\setunit*{\addcomma\space}%
	\printfield{pages}}

%Doppelpunkt nach dem letzten Autor
\renewcommand*{\labelnamepunct}{\addcolon\addspace }

%Komma an Stelle des Punktes
\renewcommand*{\newunitpunct}{\addcomma\space}

%Autoren durch Semikolon trennen
\newcommand*{\bibmultinamedelim}{\addcomma\space}% 
\newcommand*{\bibfinalnamedelim}{\addcomma\space}% 
\AtBeginBibliography{% 
  \let\multinamedelim\bibmultinamedelim 
  \let\finalnamedelim\bibfinalnamedelim 
}

%Titel nicht kursiv anzeigen 
\DeclareFieldFormat{title}{#1\isdot}






%Bib-Datei einbinden
\addbibresource{literatur/literatur.bib}

% Pfad fuer Abbildungen
\graphicspath{{./}{./abbildungen/}}

%-----------------------------------
% Weitere Ebene einfügen
\input{skripte/weitereEbene}

%-----------------------------------
% Zeilenabstand 1,5-zeilig
%-----------------------------------
\usepackage{setspace}
\onehalfspacing

%-----------------------------------
% Absätze durch eine neue Zeile
%-----------------------------------
\setlength{\parindent}{0mm}
\setlength{\parskip}{0.8em plus 0.5em minus 0.3em}

\sloppy					%Abstände variieren
\pagestyle{headings}

%-----------------------------------
% Abkürzungsverzeichnis
%-----------------------------------
\usepackage[intoc]{nomencl}
\renewcommand{\nomname}{Abkürzungsverzeichnis}
\setlength{\nomlabelwidth}{.20\textwidth}
\renewcommand{\nomlabel}[1]{\textbf{#1} \dotfill}
\setlength{\nomitemsep}{-\parsep}
\makenomenclature

\usepackage[nohyperlinks, printonlyused]{acronym}



%-----------------------------------
% Meta informationen
%-----------------------------------
%-----------------------------------
% Meta Informationen zur Arbeit
%-----------------------------------

% Autor
\newcommand{\myAutor}{Max Mustermann}

% Adresse
\newcommand{\myAdresse}{Straße 1 \\ \> \> 12345 Hamburg}

% Titel der Arbeit
\newcommand{\myTitel}{Einführung in LaTex}

% Betreuer
\newcommand{\myBetreuer}{Professor Dr. Peter Petersen}

% Lehrveranstaltung
\newcommand{\myLehrveranstaltung}{Modul Nr. 1}

% Matrikelnummer
\newcommand{\myMatrikelNr}{123456}

% Ort
\newcommand{\myOrt}{Hamburg}

% Datum der Abgabe
\newcommand{\myAbgabeDatum}{2020-01-02}

% Semesterzahl
\newcommand{\mySemesterZahl}{8}

% Name der Hochschule
\newcommand{\myHochschulName}{FOM Hochschule für Oekonomie \& Management}

% Standort der Hochschule
\newcommand{\myHochschulStandort}{Hochschulzentrum Hamburg}

% Studiengang
\newcommand{\myStudiengang}{IT Management}

% Art der Arbeit
\newcommand{\myThesisArt}{Master-Thesis}

% Zu erlangender akademische Grad
\newcommand{\myAkademischerGrad}{Master of Science (M.Sc.)}

% Firma
\newcommand{\myFirma}{Mustermann GmbH}

%-----------------------------------
% PDF Meta Daten setzen
%-----------------------------------
\hypersetup{
    pdfinfo={
        Title={\myTitel},
        Subject={\myStudiengang},
        Author={\myAutor},
        Build=1.1
    }
}

%-----------------------------------
% Umlaute in Code korrekt darstellen
% siehe auch: https://en.wikibooks.org/wiki/LaTeX/Source_Code_Listings
%-----------------------------------
\lstset{literate=
	{á}{{\'a}}1 {é}{{\'e}}1 {í}{{\'i}}1 {ó}{{\'o}}1 {ú}{{\'u}}1
	{Á}{{\'A}}1 {É}{{\'E}}1 {Í}{{\'I}}1 {Ó}{{\'O}}1 {Ú}{{\'U}}1
	{à}{{\'a}}1 {è}{{\'e}}1 {ì}{{\'i}}1 {ò}{{\'o}}1 {ù}{{\'u}}1
	{À}{{\'A}}1 {È}{{\'E}}1 {Ì}{{\'I}}1 {Ò}{{\'O}}1 {Ù}{{\'U}}1
	{ä}{{\"a}}1 {ë}{{\"e}}1 {ï}{{\"i}}1 {ö}{{\"o}}1 {ü}{{\"u}}1
	{Ä}{{\"A}}1 {Ë}{{\"E}}1 {Ï}{{\"I}}1 {Ö}{{\"O}}1 {Ü}{{\"U}}1
	{â}{{\^a}}1 {ê}{{\^e}}1 {î}{{\^i}}1 {ô}{{\^o}}1 {û}{{\^u}}1
	{Â}{{\^A}}1 {Ê}{{\^E}}1 {Î}{{\^I}}1 {Ô}{{\^O}}1 {Û}{{\^U}}1
	{œ}{{\oe}}1 {Œ}{{\OE}}1 {æ}{{\ae}}1 {Æ}{{\AE}}1 {ß}{{\ss}}1
	{ű}{{\H{u}}}1 {Ű}{{\H{U}}}1 {ő}{{\H{o}}}1 {Ő}{{\H{O}}}1
	{ç}{{\c c}}1 {Ç}{{\c C}}1 {ø}{{\o}}1 {å}{{\r a}}1 {Å}{{\r A}}1
	{€}{{\EUR}}1 {£}{{\pounds}}1
}

%-----------------------------------
% Kopfbereich / Header definieren
%-----------------------------------
\pagestyle{fancy}
\fancyhf{}
\fancyhead[C]{-\ \thepage\ -}						% Seitenzahl oben, mittg
%\fancyhead[L]{\leftmark}							% kein Footer vorhanden
\renewcommand{\headrulewidth}{0.4pt}

%----------------------------------
% Fusszeile definieren
%----------------------------------
\setlength{\footskip}{2cm}

%-----------------------------------
% Start the document here:
%-----------------------------------
\begin{document}

\pagenumbering{Roman}								% Seitennumerierung auf römisch umstellen
\renewcommand{\refname}{Literaturverzeichnis}		% "Literatur" in
%"Literaturverzeichnis" umbenennen
\newcolumntype{C}{>{\centering\arraybackslash}X}	% Neuer Tabellen-Spalten-Typ:
%Zentriert und umbrechbar

%-----------------------------------
% Titlepage
%-----------------------------------
\begin{titlepage}
	\newgeometry{left=2cm, right=2cm, top=2cm, bottom=2cm}
	\begin{center}
		\vspace{1.5cm}
			\includegraphics[width=3cm]{abbildungen/fomLogo.jpg} \\
		\vspace{0.9cm}
		\Large \textbf{\myHochschulName}\\
		\large{\myHochschulStandort}\\ 
		
		\vspace{1.7cm}
		
		\large \textbf{\myThesisArt} \\
		\small{im Studiengang \myStudiengang}
		
		\vspace{2cm}
		
		
		
		\small{zur Erlangung des Grades eines}\\
		\large{\myAkademischerGrad}\\
		% Oder für Hausarbeiten:
		%\textbf{im Rahmen der Lehrveranstaltung}\\
		%\textbf{\myLehrveranstaltung}\\
		\vspace{2cm}
		\small{über das Thema}\\
		\large \textbf{\myTitel}\\
		\vspace{1.5cm}
		\small{von}\\
		\vspace{0.5cm}
		\large{\myAutor}
		
	\end{center}
	\normalsize
	\vfill
	\begin{tabbing}
		Links \=MitteL \= MitteR \= Rechts\kill
		Erstgutachter \> \> \>\myBetreuer\\
		Matrikelnummer \> \> \> \myMatrikelNr\\
		Abgabe \> \> \> \myAbgabeDatum
	\end{tabbing}
\end{titlepage}

%-------Ende Titelseite-------------

%-----------------------------------
% Sperrvermerk
%-----------------------------------
%\input{kapitel/sperrvermerk}

%-----------------------------------
% Inhaltsverzeichnis
%-----------------------------------
\setcounter{page}{1}
\tableofcontents
\newpage

%-----------------------------------
% Abbildungsverzeichnis
%-----------------------------------
\listoffigures
\newpage
%-----------------------------------
% Tabellenverzeichnis
%-----------------------------------
\listoftables
\newpage
%-----------------------------------
% Abkürzungsverzeichnis
%-----------------------------------
\section*{Abkürzungsverzeichnis}
\addcontentsline{toc}{section}{Abkürzungsverzeichnis} 
\begin{acronym}[Bash]
	\acro{A}[A]{Akronym}
	\acro{B}[B]{B-kronym}
\end{acronym} 
\newpage
%-----------------------------------
% Seitennummerierung auf arabisch und ab 1 beginnend umstellen
%-----------------------------------
\pagenumbering{arabic}
\setcounter{page}{1}

%-----------------------------------
% Kapitel / Inhalte
%-----------------------------------
\section{Einleitung}
\subsection{Hintergrund}
Dieses ist das erste Unterkapitel und soll natürlich den Hintergrund und die Motivation klären. Gerne möchte ich mit dieser Veröffentlichung einen einfacheren und schneller Einstieg in LaTex ermöglichen, daher werden hier noch einige Inhalte näher erläutert. Das Akronym \ac{A} wird in dem Abkürzungsverzeichnis aufgenommen. Das Aknronym \ac{B} ebenfalls.


\subsection{Latex Befehle}
Hier haben wir jetzt ein weiteres Unterkapitel. Ein Text wird durch den Befehl \begin{verbatim} \textit{\enquote{ }} \end{verbatim} \textit{\enquote{kursiv dargestellt und erhält Gänsefüßchen.}} Dabei dürfen nach dem Text die schließenden  \}\} nicht fehlen.

Der \textbf{fette Text} wird übrigens mit nachfolgenden Befehl dargestellt:
\begin{verbatim}
\textbf{fette Text}
\end{verbatim}

Wenn man das verstanden hat, ist das schon die halbe Miete. In LaTex werden alle Befehle durch diese Konstrukt beschrieben. Weiterhin können noch Parameter übergeben werden. Das Beispiel eines Zitats stellt dies sehr gut dar:
\begin{verbatim}
\fullfootcite[Vgl. ][S. 7]{ITEMO.2017} 
\end{verbatim}
Der Zitierbefehl übergibt einerseits die Quelle, aber auch die Texte "Vgl." und "S. 7". Wenn man Lust hat eigene Befehle zu erweitern, ist dies durch die Parameterübergabe sehr gut möglich.\fullfootcite[Vgl. ][S. 1]{ABC.2013} Dazu würde ich dann an die einschlägigen Seiten noch einmal verweisen.\blogfootcite[Vgl. ][]{ABC.2017} 



\subsection{Aufzählungen}
Das dritte Unterkapitel beschäftigt sich nun mit der Aufzählung. Aufzählungen können unterschiedlich genutzt werden.
\begin{itemize}
	\item Erste Aufzählung
	\item Zweite Aufzählung
	\item Dritte Aufzählung	
\end{itemize}

Eine Aufzählung kann mit folgendem Befehl erfolgen:
\begin{verbatim}
\begin{itemize}
\item Erste Aufzählung
\item Zweite Aufzählung
\item Dritte Aufzählung	
\end{itemize}
\end{verbatim}

Wenn man jedoch eine Nummerierung der einzelnen Punkte erhalten möchte
\begin{enumerate}
	\item Erste Aufzählung
	\item Zweite Aufzählung
	\item Dritte Aufzählung	
\end{enumerate}
ist der nachfolgende Befehl zu wählen:
\begin{verbatim}
\begin{enumerate}
\item Erste Aufzählung
\item Zweite Aufzählung
\item Dritte Aufzählung	
\end{enumerate}
\end{verbatim}

\section{Kapitel in Latex}
Dies ist ein Hauptkapitel. der Befehl dazu lautet:
\begin{verbatim}
\section{Kapitel in Latex}
\end{verbatim}

\subsection{Unterkapitel}
Das Unterkapitel ist relativ einfach zu bestimmen. Man beschreibt einfach das Unter- oder auf Sub-Kapitel.
\begin{verbatim}
\subsection{Unterkapitel}
\end{verbatim}

\subsubsection{Unter-Unterkapitel}
Das Unter-Unter-Kapitel ist in LaTex-Sprache einfach das Sub-Sub-Kapitel:
\begin{verbatim}
\subsubsection{Unter-Unterkapitel}
\end{verbatim}

\subsection{Zweites Unterkapitel}
Wir wollen also nun weter zum nächsten Kapitel übergehen.












 
% !TeX encoding = UTF-8
\newpage
\section{Begriffsbestimmung}
	Dies ist ein weitere Kapitel.



		
	
\newpage
\section{Tabellen und Grafiken}
	Jede wissenschaftliche Arbeit soll mit grafischen Elementen ergänzt werden. Dieses Kapitel beschäftigt sich mit dem Einfügen von Grafiken und Tabellen:
	
	\begin{figure}[H]
		\caption{Blauer Kasten}
		\includegraphics[width=\textwidth, frame]{abbildungen/Grafik.png}
		\label{abb:externintern} 
		
		\text{Quelle: Eigene Abbildung}
	\end{figure}


	Die Grafik wird mit folgendem Befehl integriert:
	\begin{verbatim}
	\begin{figure}[H]
		\caption{Blauer Kasten}
		\includegraphics[width=\textwidth, frame]{abbildungen/Grafik.png}
		\label{abb:externintern} 
		
		\text{Quelle: Eigene Abbildung}
	\end{figure}
	\end{verbatim}
	
	
	Eine Tabelle kann für verschiedene Inhalte verwendet werden und ist nicht so wirklich schwierig einzubinden. Hierzu sollte man am besten
			\begin{table}[H]
		\caption{Titel der Tabelle}
		\begin{tabular}[ht]{|c|c|} \hline
			Spalte 1 & Spalte 2   \\ \hline
			A1 &  B1 \\ \hline
			A2 &  B2 \\ \hline 
			A3 &  B3 \\ \hline 
			A4 &  B4 \\ \hline 
			A5 &  B5 \\ \hline 
		\end{tabular} \\
		
		\text{Quelle: Eigene Darstellung}
		\label{tbl:tabelle2}
	\end{table}

	\begin{verbatim}
	\begin{table}[H]
		\caption{Übersicht der Blog-Posts auf http://devops4itsm.de}
		\begin{tabular}[ht]{|c|c|} \hline
			Spalte 1 & Spalte 2   \\ \hline
			A1 &  B1 \\ \hline
			A2 &  B2 \\ \hline 
			A3 &  B3 \\ \hline 
			A4 &  B4 \\ \hline 
			A5 &  B5 \\ \hline 
		\end{tabular} \\
	
		\text{Quelle: Eigene Darstellung}
		\label{tbl:tabelle2}
	\end{table}
	\end{verbatim}
	
\section{Ende}
	Das war es dann auch schon. Ich hoffe, dass dieses ein guter Einstieg ist und ein einfacher Einstieg in LaTex gelingt.


	
	
	


%-----------------------------------
% Literaturverzeichnis
%-----------------------------------
\newpage
%\addcontentsline{toc}{section}{Literatur}

\pagenumbering{Roman} %Zähler wieder römisch ausgeben
\setcounter{page}{6}  %Zähler manuell hochsetzen

%Definieren der Dokumententypen, die verwendet werden dürfen / Interviewmaterial sollte ausgeklammert werden.
\defbibfilter{BuchArtikel}{% 
	( type=book or type=article or type=online or type=incollection or type=collection  ) 
}

%Interviews werden hier durch PDF-Dateien eingebunden. Hierzu wird eine weitere tex-Datei aufgerufen.
%\pagenumbering{Roman}								% Seitennumerierung auf römisch umstellen
\newpage
%\setcounter{page}{1}
\section*{Anhang} 
\addcontentsline{toc}{section}{Anhang} 

%\addcontentsline{toc}{section}{Anhang 1: Interview 1} %Die erste Seite wird eingebunden
\includepdf[pages=1, link, linkname=Interview1, noautoscale=true, scale=0.9, pagecommand={\section*{Anhang 1: Interview 1}\thispagestyle{empty}}]{interviews/interview1.pdf} 
%% Die restlichen Seiten werden eingebunden.
\includepdf[pages=2-last, link, linkname=Interview1, noautoscale=true, scale=0.9, pagecommand={\thispagestyle{empty}}]{interviews/interview1.pdf} 

%\addcontentsline{toc}{section}{Anhang 7: Code-System} 
\includepdf[pages=1, link, linkname=CodeSystem, noautoscale=true, scale=0.9,
pagecommand={\section*{Anhang 2: Code-System}\thispagestyle{empty}}]{anhang/CodeUebersicht.pdf} 
\label{anhang:code}
% \includepdf[pages=2-last, link, linkname=CodeSystem, noautoscale=true, scale=0.9, % % pagecommand={\thispagestyle{empty}}]{anhang/CodeUebersicht.pdf} 






%\printbibliography
\printbibliography[% 
heading=bibintoc, 
filter=BuchArtikel, 
title={Literaturverzeichnis} 
]
% Alternative Darstellung:
% Literaturverzeichnis nach Typ (@book, @arcticle ...) sortiert.
% Dazu die Zeile (\printbibliography) auskommentieren und folgenden code verwenden:
%\printbibheading
%\printbibliography[type=article,heading=subbibliography,title={Artikel}]
%\printbibliography[type=book,heading=subbibliography,title={Bücher}]
%\printbibliography[type=online,heading=subbibliography,title={Webseiten}]
%\printbibliography[type=misc,heading=subbibliography,title={zus. Beiträge}]
%\printbibliography[type=incollection,heading=subbibliography,title={Beiträge in}]
%\printbibliography[type=collection,heading=subbibliography,title={Sammelwerke}]



% Hier wird die wichtige Erklärung erzeugt.
\input{kapitel/erklaerung}
\end{document}
