\newpage
\section{Tabellen und Grafiken}
	Jede wissenschaftliche Arbeit soll mit grafischen Elementen ergänzt werden. Dieses Kapitel beschäftigt sich mit dem Einfügen von Grafiken und Tabellen:
	
	\begin{figure}[H]
		\caption{Blauer Kasten}
		\includegraphics[width=\textwidth, frame]{abbildungen/Grafik.png}
		\label{abb:externintern} 
		
		\text{Quelle: Eigene Abbildung}
	\end{figure}


	Die Grafik wird mit folgendem Befehl integriert:
	\begin{verbatim}
	\begin{figure}[H]
		\caption{Blauer Kasten}
		\includegraphics[width=\textwidth, frame]{abbildungen/Grafik.png}
		\label{abb:externintern} 
		
		\text{Quelle: Eigene Abbildung}
	\end{figure}
	\end{verbatim}
	
	
	Eine Tabelle kann für verschiedene Inhalte verwendet werden und ist nicht so wirklich schwierig einzubinden. Hierzu sollte man am besten
			\begin{table}[H]
		\caption{Titel der Tabelle}
		\begin{tabular}[ht]{|c|c|} \hline
			Spalte 1 & Spalte 2   \\ \hline
			A1 &  B1 \\ \hline
			A2 &  B2 \\ \hline 
			A3 &  B3 \\ \hline 
			A4 &  B4 \\ \hline 
			A5 &  B5 \\ \hline 
		\end{tabular} \\
		
		\text{Quelle: Eigene Darstellung}
		\label{tbl:tabelle2}
	\end{table}

	\begin{verbatim}
	\begin{table}[H]
		\caption{Übersicht der Blog-Posts auf http://devops4itsm.de}
		\begin{tabular}[ht]{|c|c|} \hline
			Spalte 1 & Spalte 2   \\ \hline
			A1 &  B1 \\ \hline
			A2 &  B2 \\ \hline 
			A3 &  B3 \\ \hline 
			A4 &  B4 \\ \hline 
			A5 &  B5 \\ \hline 
		\end{tabular} \\
	
		\text{Quelle: Eigene Darstellung}
		\label{tbl:tabelle2}
	\end{table}
	\end{verbatim}
	
\section{Ende}
	Das war es dann auch schon. Ich hoffe, dass dieses ein guter Einstieg ist und ein einfacher Einstieg in LaTex gelingt.


	
	
	
